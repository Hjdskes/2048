\documentclass[a4paper,11pt,report]{scrartcl}
\usepackage[dutch]{babel}
\usepackage[T1]{fontenc}
\usepackage[utf8]{inputenc}
\usepackage{lmodern}
\usepackage{amssymb}
\usepackage{color}
\usepackage{graphicx}
\usepackage{mathtools}
\DeclareGraphicsExtensions{.pdf,.png,.jpg}
\newcommand{\tab}{\hspace*{2em}}

\title{\huge\textbf{Deliverable 4}}
\author{Piet van Agtmaal 4321278\\
	    Jochem Heijltjes 1534041\\
		Arthur Hovanesyan 4322711\\
		Paul Bakker 4326091\\
		Jente Hidskes 4335732
	   }

\begin{document}
\begin{titlepage}
\maketitle
\thispagestyle{empty} %geen page numbering op opening pagina
\end{titlepage}

\newpage\section{Preamble}

We are five Computer Science students at the Delft University of Technology,
The Netherlands. We created a clone of the original 2048 game for the course
Software Engineering Methods.\\

The goal of this course is to teach us how to develop software by applying the most appropriate software engineering practices, given the context of development.\\

We would like to thank our teacher Dr. Bachelli and our Teaching Assistants Moritz and Aimee for guiding us through this project.\\

Thank you for reading this report!\\

\newpage\section{Introduction}

2048 is a very popular game created by by Gabriele Cirulli, based on 1024 by
Veewo Studio and conceptually similar to Threes by Asher Vollmer.\\

For this project we were asked to create a clone of the game 2048 in the Object-Oriented programming language Java, so that's what we did!\\

We didn't only make a clone of the original 2048 game, we also added
multiplayer functionality, automatic solvers, playing against the computer. 
Now you can challenge your computer, your neighbours, your
coworkers, your kids or even the Queen of England to play a game of 2048!\\

This document tells you everything you need to know about our project. Technical aspects of
the game are covered in here as well.\\

Setting up a multiplayer game is really straightforward, however. Just make sure incoming connections to port TCP/2048 reach the hosting machine! Yes, port 2048! Who would have thunk* it? More details are up ahead!\\

Good luck and have fun!\\

\textbf{P.S.:} We, the developers, are not reliable in case of frustration. ;)\\

\textbf{*} See: http://english.stackexchange.com/questions/55577/proper-usage-of-the-word-thunk

\newpage\section{How to play 2048}

This section briefly describes how to play 2048 and provides information on
the functionality it has, such as playing the game alone, with friends and how
to use the logging features.\\

\subsection{Singleplayer game}
After starting the application you will see the main menu. In the main menu,
click the \texttt{Singleplayer} button to start your singleplayer game.\\

You move the tiles with the arrow keys. Each time two tiles with the same
number collide, the numbers are added and the two tiles merge. Your goal is to
reach the 2048 tile!\\

To return to the menu, click the \texttt{Escape} button any time. Don't worry,
your current game will be saved for you! (This also applies to closing the
game!).

\subsection{Multiplayer game}
The multiplayer version is identical to the singleplayer, except here you will
compete against a friend, colleague, coworker or your worst enemy over LAN or
the internet. Your opponent does not have to be in the same room with you; they
can even be on the other side of the planet and you can still kick their
asses!\\

Your goal is to reach the 2048 tile \textit{before} your opponent does. In case you
are unable to reach the 2048 tile (e.g., because you lost), your opponent
automatically wins.\\

We will now briefly explain how to connect to eachother. Please refer to the
documentation of your networking equipment or software in case you experience
networking problems.

\subsubsection{Joining a game}
To connect to another player, choose the \texttt{Join a game} option in the
main menu. The application will try to connect to the remote address you
entered, on port 2048, using TCP.

\subsubsection{Hosting a game}
To have another player connect to you, choose the \texttt{Host a game} option
in the main menu. The application will bind to port 2048/TCP on all the
system's network interfaces. In case you wish to play over the internet,
please make sure connections on this port are forwarded to your local address
on your NAT device. Consult the manual of your network products for more 
information.


\subsection{Challenging your computer}



Lorem ipsum dolor sit amet, consectetur adipiscing elit. Morbi fermentum quam vel enim consectetur sollicitudin. Aenean eget justo at diam condimentum blandit. Vestibulum massa lectus, tincidunt tincidunt ligula ac, venenatis imperdiet libero. Quisque at sem risus. Vestibulum ante ipsum primis in faucibus orci luctus et ultrices posuere cubilia Curae; Phasellus efficitur augue felis, vitae tristique elit laoreet in. Suspendisse non elit dapibus, pharetra velit et, scelerisque tellus.

Pellentesque eu mattis nibh. Aliquam quis ultrices nibh. Praesent gravida et orci in tincidunt. Suspendisse vel elementum mauris. Aliquam odio eros, cursus nec luctus id, elementum at orci. Integer malesuada ipsum condimentum rhoncus vestibulum. Phasellus cursus non ligula at semper.

Donec molestie sed sapien eu malesuada. Nulla at faucibus tellus, molestie placerat neque. Nulla non viverra eros, a pretium nisi. Integer scelerisque pellentesque massa a consectetur. Sed sodales urna in sem laoreet ultrices. Fusce vitae est mattis, aliquet ipsum non, faucibus tortor. Pellentesque velit leo, molestie in pretium at, congue quis urna. Nulla tincidunt iaculis neque in ullamcorper. In lacinia eu libero vel tincidunt. Sed dolor nibh, efficitur sit amet lectus quis, commodo iaculis mi. Cras iaculis eget tortor ut elementum. Cras magna nibh, maximus a pharetra in, convallis quis felis. Aenean finibus metus eget lectus aliquam, et semper nisi interdum. Nulla pellentesque risus ut metus ultrices interdum. Vivamus convallis malesuada mauris non scelerisque. Aenean ante risus, imperdiet eu dui nec, auctor luctus lacus.

Integer rutrum, velit ac laoreet scelerisque, tortor felis fringilla dolor, eu faucibus risus sapien et odio. Duis sit amet urna tristique, iaculis purus id, sodales tortor. Pellentesque nibh velit, pulvinar vitae velit et, finibus suscipit leo. Morbi ac quam ex. Sed eu pharetra enim. Cras id lobortis elit. Phasellus pulvinar magna sit amet malesuada mollis. Morbi nec enim sed magna fringilla suscipit. Sed malesuada elementum quam id cursus. Ut maximus est vel arcu rutrum, at eleifend justo ullamcorper. Vivamus malesuada urna sit amet rutrum porttitor. Maecenas nulla ipsum, tincidunt at dui sed, convallis tincidunt lorem. Sed placerat tellus in est molestie fermentum. Duis tincidunt volutpat consequat. Morbi sodales elementum justo, eget iaculis lorem.

Sed lorem quam, fringilla id gravida sit amet, feugiat ut lectus. Mauris accumsan tellus sed sapien bibendum fringilla. Donec rhoncus tellus vel ipsum scelerisque, ac sollicitudin ligula tincidunt. Integer a dolor augue. Donec vehicula augue id sapien cursus semper. Praesent dui eros, lobortis non feugiat et, sagittis sed mauris. Sed velit quam, dignissim et nibh rhoncus, accumsan ullamcorper lectus. Vestibulum elementum, neque rhoncus egestas viverra, ex arcu vulputate libero, sit amet consectetur lorem orci ac risus. Vivamus feugiat, neque tincidunt bibendum aliquam, ante tortor congue leo, quis volutpat augue mauris a libero. Vivamus arcu lectus, ornare id augue in, vehicula luctus felis. Ut vestibulum tortor tortor, sit amet maximus enim malesuada quis. Cras ac tempor leo, sed faucibus neque. Ut dictum odio lectus, ut vehicula urna maximus non. 

\subsection{Logging}
The game supports several commandline arguments for logging.\\

By default, the application will log to the standard output, using the
\texttt{ALL} logging level. If enabled, however, errors will be logged to
\texttt{stderr}. The logging level can also be adjusted.\\

The supported arguments are:
\begin{verbatim}
$ jarfile.jar [logLevel] [file]
\end{verbatim}
or, otherwise:
\begin{verbatim}
$ Launcher.java [logLevel] [file]
\end{verbatim}
Both of these fields are parsed case-insensitively.\\

Two examples:
\begin{verbatim}
$ Launcher.java debug
\end{verbatim}
will run the game and log all debug and info messages. 
\begin{verbatim}
$ Launcher.java error file
\end{verbatim}
will run the game and log all debug, error and info messages to the system's
output streams (\texttt{stdout} and \texttt{stderr}) and will write them to a
new file as well.\\

Please see the corresponding section below for more information on the possible
arguments:\\

\textbf{logLevel} can be one of the following:
\begin{description}
	\item[all] logs all messages;
	\item[info] logs info messages only;
	\item[error] log error messages and info messages;
	\item[debug] log debug, error and info messages;
	\item[none] disables logging.
\end{description}

\textbf{file}

Setting the \texttt{file} flag will write all messages of the previously set
logging level to a file. By default, a new file with the format
\texttt{2048\_YYYYMMDD\_HHmmss.log} will be created, where
\texttt{YYYMMMDDD\_HHmmss} is the time of application start.

\newpage\section{Test report}

In this section we will explain how we tested our game. We will start by
explaining how often we tested our game. Afterwards, we will explain what
kinds of testing we have done. Lastly, we will present the results of the
testing procedure.

\subsection{Test frequency}
In this section we will discuss how frequently we tested our game. Due to the
design patters we implemented, testing was essential. We tested using our
unit tests (locally and on Devhub as well) and visual tests. Implementing
iterators caused some problems that only arised during our visual tests. These
were, however, resolved immediately.

\subsection{Testing methods}
Visual tests involved actually playing the game and analyzing logging output
manually. Unit tests simply check object properties with certain input.

\subsection{Test results}
EclEmma is the tool we used for analyzing and measuring our test coverage.
As before, we analyzed our entire project using three different metrics: line,
branch and instruction coverage.\\

The results are as follows:
\begin{description}
	\item Line coverage: 72.3\%
	\item Branch coverage: 62.0\%
	\item Instruction coverage: 68.8\%
\end{description}
As with previous deliverables, we faced the same issues with code that requires
graphically rendering our game. 

\subsection{Conclusion}
Although our test results are lower than we had planned to achieve, we believe
our project has again been tested sufficiently.


\end{document}
